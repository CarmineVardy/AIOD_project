\documentclass[journal, a4paper, onecolumn]{IEEEtran}

% some very useful LaTeX packages include:

%\usepackage{cite}      % Written by Donald Arseneau
                        % V1.6 and later of IEEEtran pre-defines the format
                        % of the cite.sty package \cite{} output to follow
                        % that of IEEE. Loading the cite package will
                        % result in citation numbers being automatically
                        % sorted and properly "ranged". i.e.,
                        % [1], [9], [2], [7], [5], [6]
                        % (without using cite.sty)
                        % will become:
                        % [1], [2], [5]--[7], [9] (using cite.sty)
                        % cite.sty's \cite will automatically add leading
                        % space, if needed. Use cite.sty's noadjust option
                        % (cite.sty V3.8 and later) if you want to turn this
                        % off. cite.sty is already installed on most LaTeX
                        % systems. The latest version can be obtained at:
                        % http://www.ctan.org/tex-archive/macros/latex/contrib/supported/cite/

\usepackage{graphicx}   % Written by David Carlisle and Sebastian Rahtz
                        % Required if you want graphics, photos, etc.
                        % graphicx.sty is already installed on most LaTeX
                        % systems. The latest version and documentation can
                        % be obtained at:
                        % http://www.ctan.org/tex-archive/macros/latex/required/graphics/
                        % Another good source of documentation is "Using
                        % Imported Graphics in LaTeX2e" by Keith Reckdahl
                        % which can be found as esplatex.ps and epslatex.pdf
                        % at: http://www.ctan.org/tex-archive/info/

%\usepackage{psfrag}    % Written by Craig Barratt, Michael C. Grant,
                        % and David Carlisle
                        % This package allows you to substitute LaTeX
                        % commands for text in imported EPS graphic files.
                        % In this way, LaTeX symbols can be placed into
                        % graphics that have been generated by other
                        % applications. You must use latex->dvips->ps2pdf
                        % workflow (not direct pdf output from pdflatex) if
                        % you wish to use this capability because it works
                        % via some PostScript tricks. Alternatively, the
                        % graphics could be processed as separate files via
                        % psfrag and dvips, then converted to PDF for
                        % inclusion in the main file which uses pdflatex.
                        % Docs are in "The PSfrag System" by Michael C. Grant
                        % and David Carlisle. There is also some information
                        % about using psfrag in "Using Imported Graphics in
                        % LaTeX2e" by Keith Reckdahl which documents the
                        % graphicx package (see above). The psfrag package
                        % and documentation can be obtained at:
                        % http://www.ctan.org/tex-archive/macros/latex/contrib/supported/psfrag/

%\usepackage{subfigure} % Written by Steven Douglas Cochran
                        % This package makes it easy to put subfigures
                        % in your figures. i.e., "figure 1a and 1b"
                        % Docs are in "Using Imported Graphics in LaTeX2e"
                        % by Keith Reckdahl which also documents the graphicx
                        % package (see above). subfigure.sty is already
                        % installed on most LaTeX systems. The latest version
                        % and documentation can be obtained at:
                        % http://www.ctan.org/tex-archive/macros/latex/contrib/supported/subfigure/

\usepackage{url}        % Written by Donald Arseneau
                        % Provides better support for handling and breaking
                        % URLs. url.sty is already installed on most LaTeX
                        % systems. The latest version can be obtained at:
                        % http://www.ctan.org/tex-archive/macros/latex/contrib/other/misc/
                        % Read the url.sty source comments for usage information.

%\usepackage{stfloats}  % Written by Sigitas Tolusis
                        % Gives LaTeX2e the ability to do double column
                        % floats at the bottom of the page as well as the top.
                        % (e.g., "\begin{figure*}[!b]" is not normally
                        % possible in LaTeX2e). This is an invasive package
                        % which rewrites many portions of the LaTeX2e output
                        % routines. It may not work with other packages that
                        % modify the LaTeX2e output routine and/or with other
                        % versions of LaTeX. The latest version and
                        % documentation can be obtained at:
                        % http://www.ctan.org/tex-archive/macros/latex/contrib/supported/sttools/
                        % Documentation is contained in the stfloats.sty
                        % comments as well as in the presfull.pdf file.
                        % Do not use the stfloats baselinefloat ability as
                        % IEEE does not allow \baselineskip to stretch.
                        % Authors submitting work to the IEEE should note
                        % that IEEE rarely uses double column equations and
                        % that authors should try to avoid such use.
                        % Do not be tempted to use the cuted.sty or
                        % midfloat.sty package (by the same author) as IEEE
                        % does not format its papers in such ways.

\usepackage{amsmath}    % From the American Mathematical Society
                        % A popular package that provides many helpful commands
                        % for dealing with mathematics. Note that the AMSmath
                        % package sets \interdisplaylinepenalty to 10000 thus
                        % preventing page breaks from occurring within multiline
                        % equations. Use:
%\interdisplaylinepenalty=2500
                        % after loading amsmath to restore such page breaks
                        % as IEEEtran.cls normally does. amsmath.sty is already
                        % installed on most LaTeX systems. The latest version
                        % and documentation can be obtained at:
                        % http://www.ctan.org/tex-archive/macros/latex/required/amslatex/math/

% Other popular packages for formatting tables and equations include:

%\usepackage{array}
% Frank Mittelbach's and David Carlisle's array.sty which improves the
% LaTeX2e array and tabular environments to provide better appearances and
% additional user controls. array.sty is already installed on most systems.
% The latest version and documentation can be obtained at:
% http://www.ctan.org/tex-archive/macros/latex/required/tools/

% V1.6 of IEEEtran contains the IEEEeqnarray family of commands that can
% be used to generate multiline equations as well as matrices, tables, etc.

% Also of notable interest:
% Scott Pakin's eqparbox package for creating (automatically sized) equal
% width boxes. Available:
% http://www.ctan.org/tex-archive/macros/latex/contrib/supported/eqparbox/

% *** Do not adjust lengths that control margins, column widths, etc. ***
% *** Do not use packages that alter fonts (such as pslatex).         ***
% There should be no need to do such things with IEEEtran.cls V1.6 and later.

%\usepackage{hyperref}
% To insert hyperlinks in the document

\usepackage{microtype}

\renewcommand{\thesection}{\arabic{section}}
\renewcommand{\thesubsection}{\thesection.\arabic{subsection}}
\renewcommand{\thesubsubsection}{\thesubsection.\arabic{subsubsection}}

\makeatletter
\def\thesectiondis{\thesection}
\def\thesubsectiondis{\thesubsection}
\def\thesubsubsectiondis{\thesubsubsection}
\makeatother

\usepackage[justification=centering]{caption}

\usepackage{titlesec}
\titleformat{\section}{\normalfont\Large\bfseries}{\thesection}{1em}{}
\titleformat{\subsection}{\normalfont\large\bfseries}{\thesubsection}{1em}{}

\usepackage{fancyhdr}
\usepackage{float}
\pagestyle{fancy}
\fancyhf{}
\renewcommand{\headrulewidth}{0pt}

\fancyhead[L]{\small Information Engineering for Digital Medicine - Artificial Intelligence for Omics Data Analysis Course 2025-2026}

\fancyfoot[R]{\thepage}


% Your document starts here!
\begin{document}

\begin{titlepage}
    \centering
    \vspace*{\fill}
    {\Huge \textbf{Project Report} \par}
    \vspace{1.5cm}
    {\Large
    Marco Savastano \\
    Carmine Vardaro \par}
    \vspace{2cm}
    {\large Information Engineering for Digital Medicine \\ Artificial Intelligence for Omics Data Analysis Course 2025-2026 \par}
    \vspace*{\fill}
\end{titlepage}

\newpage
\thispagestyle{fancy}
\setcounter{page}{1}
\tableofcontents
\newpage

\thispagestyle{fancy}

\begin{abstract}
    The short abstract (50-80 words) is intended to give the reader an overview of the work.
\end{abstract}

\section{Introduction}

\subsection{Background Clinico}
Breve panoramica sulla patologia (CHD) e importanza di trovare nuovi biomarcatori non invasivi.

\subsection{La Metabolomica Untargeted}
Perché la LC-MS è la scelta giusta qui (visione olistica del fenotipo).

\subsection{Problematiche Aperte}
Qui introduci il "problema" del tuo progetto: la complessità dei dati, la necessità di integrare due modi di ionizzazione (ESI+/ESI-) e la scelta del miglior metodo di preprocessing (non esiste una "ricetta unica").

\subsection{Scopo del Lavoro}
Valutare e ottimizzare un workflow chemiometrico (dal preprocessing alla Data Fusion) per distinguere soggetti Sani vs Patologici e identificare le feature biologicamente rilevanti.

\section{Materials and Methods}

\subsection{Descrizione del Dataset}
The dataset is composed of metabolomic data acquired by LC–MS in both positive  (ESI+)  and  negative  (ESI-)  ionisation  mode.

There are no missing values and zeros, due to an value inputation phase already done focused on the replacement with one-fifth of the minimum value recorded in the dataset for that molecule.
\cite{CHD}

The following tables give an overview on the dataset composition, for both ESI+ and ESI- with the corresponding values:

	\begin{table}[H]
		\begin{center}
			\caption{ESI- Dataset Distribution and Characteristics}
			\label{tab:esi-dataset_details}
			\begin{tabular}{|l|c|}
				\hline
				\textbf{DESCRIPTION} & \textbf{VALUE} \\
				\hline
				Total Samples & 219 \\
				\hline
				Total Features (Metabolites) & 52 \\
				\hline
				Class Count: CTRL & 107 \\
				\hline
				Class Count: CHD & 104 \\
				\hline
				Class Count: QC & 8 \\
				\hline
				Samples with suffix '\_00' & 28 \\
				\hline
				Samples with suffix '\_01' (Tech Replicate) & 14 \\
				\hline
				Samples without suffix & 177 \\
				\hline
				Estimated Unique Biological Samples & 205 \\
				\hline
				CTRL - Biological Samples & 100 \\
				\hline
				CTRL - Technical Replicates & 7 \\
				\hline
				CHD - Biological Samples & 97 \\
				\hline
				CHD - Technical Replicates & 7 \\
				\hline
				QC - Total Samples & 8 \\
				\hline
				Negative Values Present & No \\
				\hline
			\end{tabular}
		\end{center}
	\end{table}
	
	
	\begin{table}[H]
		\begin{center}
			\caption{ESI+ Dataset Distribution and Characteristics}
			\label{tab:esi+dataset_details}
			\begin{tabular}{|l|c|}
				\hline
				\textbf{DESCRIPTION} & \textbf{VALUE} \\
				\hline
				Total Samples & 219 \\
				\hline
				Total Features (Metabolites) & 98 \\
				\hline
				Class Count: CTRL & 107 \\
				\hline
				Class Count: CHD & 104 \\
				\hline
				Class Count: QC & 8 \\
				\hline
				Samples with suffix '\_00' & 28 \\
				\hline
				Samples with suffix '\_01' (Tech Replicate) & 14 \\
				\hline
				Samples without suffix & 177 \\
				\hline
				Estimated Unique Biological Samples & 205 \\
				\hline
				CTRL - Biological Samples & 100 \\
				\hline
				CTRL - Technical Replicates & 7 \\
				\hline
				CHD - Biological Samples & 97 \\
				\hline
				CHD - Technical Replicates & 7 \\
				\hline
				QC - Total Samples & 8 \\
				\hline
				Negative Values Present & No \\
				\hline
			\end{tabular}
		\end{center}
	\end{table}


\subsection{Quality Assessment \& Data Cleaning}
Descrizione dei QC e dei Replicati Tecnici.
\textbf{Nota:} Spiegare chiaramente che QC e duplicati sono stati utilizzati per una valutazione iniziale della stabilità ma rimossi dal dataset finale di modeling (es. numero esiguo, impossibilità di mediare).

\subsection{Strategie di Pre-processing (Iterativo)}
Elenco delle tecniche testate: Normalizzazione, Trasformazione Logaritmica, Scaling (Autoscaling).

\subsection{Anomaly Detection}
Descrizione del metodo usato per identificare e rimuovere gli outlier (fondamentale per la pulizia del dato).

\subsection{Feature Selection}
Metodi statistici/algoritmici utilizzati per ridurre la dimensionalità e rimuovere il rumore prima del modeling.

\subsection{Strategie di Data Fusion}
Definizione degli approcci: Low-Level (concatenazione semplice) vs High-Level (o altri approcci). L'obiettivo è sfruttare la complementarità ESI+/ESI-.

\subsection{Analisi Statistica e Machine Learning}
\begin{itemize}
    \item \textbf{Unsupervised:} PCA (per l'esplorazione).
    \item \textbf{Supervised:} PLS-DA, SVM, Random Forest, Logistic Regression.
    \item \textbf{Validazione:} Descrizione rigorosa dello split Training Set vs Validation Set (o Cross-Validation) per evitare l'overfitting.
\end{itemize}

\subsection{Stack Tecnologico}
Breve paragrafo sulle librerie Python utilizzate (Pandas, Scikit-learn, ecc.) per garantire la riproducibilità.

\section{Results and Discussion}

\subsection{Valutazione dell'Analisi Esplorativa (PCA)}
Visualizzazione dei dati ESI+ e ESI- separati. Valutazione degli Outlier (prima e dopo la rimozione). Confronto delle tecniche di scaling (es. efficacia dell'autoscaling).

\subsection{Performance dei Modelli su Singoli Dataset (ESI+ / ESI-)}
Confronto delle metriche (Accuratezza, Specificità, Sensibilità) tra PLS-DA, SVM, RF, LR. Quale modello performa meglio sui dati positivi? E sui negativi?

\subsection{Risultati della Data Fusion}
La fusione dei dati ha migliorato la classificazione rispetto ai dataset singoli?

\subsection{Interpretabilità e Biomarcatori (Feature Importance)}
Analisi delle Feature Importances e Analisi Univariata (Volcano Plot). Identificazione/interpretazione biologica dei top-metaboliti.


\section{Conclusions}
Sintesi del miglior workflow identificato. Considerazioni sull'interpretabilità biologica e limiti dello studio (es. numero di campioni, assenza di validazione esterna).


% Now we need a bibliography:
\begin{thebibliography}{5}

	%Each item starts with a \bibitem{reference} command and the details thereafter.
	\bibitem{CHD} % Transaction paper
	Mires, Stuart, et al. "Plasma metabolomic and lipidomic profiles accurately classify mothers of children with congenital heart disease: an observational study." Metabolomics 20.4 (2024): 70.

\end{thebibliography}

	% This is how you define a table: the [!hbt] means that LaTeX is forced (by the !) to place the table exactly here (by h), or if that doesnt work because of a pagebreak or so, it tries to place the table to the bottom of the page (by b) or the top (by t).
	\begin{table}[!hbt]
		% Center the table
		\begin{center}
		% Title of the table
		\caption{Simulation Parameters}
		\label{tab:simParameters}
		% Table itself: here we have two columns which are centered and have lines to the left, right and in the middle: |c|c|
		\begin{tabular}{|c|c|}
			% To create a horizontal line, type \hline
			\hline
			% To end a column type &
			% For a linebreak type \\
			Information message length & $k=16000$ bit \\
			\hline
			Radio segment size & $b=160$ bit \\
			\hline
			Rate of component codes & $R_{cc}=1/3$\\
			\hline
			Polynomial of component encoders & $[1 , 33/37 , 25/37]_8$\\
			\hline
		\end{tabular}
		\end{center}
	\end{table}

	% If you have questions about how to write mathematical formulas in LaTeX, please read a LaTeX book or the 'Not So Short Introduction to LaTeX': tobi.oetiker.ch/lshort/lshort.pdf

	% This is how you include a eps figure in your document. LaTeX only accepts EPS or TIFF files.
	\begin{figure}[!hbt]
		% Center the figure.
		\begin{center}
		% Include the eps file, scale it such that it's width equals the column width. You can also put width=8cm for example...
		\includegraphics[width=0.5\columnwidth]{../images/BIPLOT_2024_Metabolomica_Neg.png}
		% Create a subtitle for the figure.
		\caption{Simulation results on the AWGN channel. Average throughput $k/n$ vs $E_s/N_0$.}
		% Define the label of the figure. It's good to use 'fig:title', so you know that the label belongs to a figure.
		\label{fig:tf_plot}
		\end{center}
	\end{figure}

% Your document ends here!
\end{document}